% \documentclass{template/openetcs_article}

% \usepackage{graphicx}

% %\usepackage[top=3cm, bottom=2.5cm, left=3cm, right=2.5cm] {geometry}
% \usepackage {bsymb,b2latex}
% \usepackage{b2latex}
% \usepackage{fancyhdr}
% %\usepackge{lastpage}%
% %\usepackge{color}
% \graphicspath{{./template/}{.}{./images/}}
% %\usepackage[utf8]{inputenc}

%---------------------------------------------------------

\documentclass{template/openetcs_article}
% Use the option "nocc" if the document is not licensed under Creative Commons
%\documentclass[nocc]{template/openetcs_article}
\usepackage {bsymb,b2latex}
\usepackage{lipsum,url,color}
\graphicspath{{./template/}{.}{./images/}}
\begin{document}
\frontmatter
\project{openETCS}

%Please do not change anything above this line
%============================
% The document metadata is defined below

%assign a report number here
\reportnum{OETCS}

%define your workpackage here
\wp{Work-Package 7: ``Toolchain''}


\newcommand{\true}{\ensuremath{true}}
\newcommand{\btext}[1]{{\it #1}}
\newcommand{\bvar}[1]{\btext{#1}}
\newcommand{\bevent}[1]{\btext{#1}}
\newcommand{\binv}[1]{\btext{#1}}
\newcommand{\bconst}[1]{\btext{#1}}
\newcommand{\bparam}[1]{\btext{#1}}
\newcommand{\bfunc}[1]{\btext{#1}}
\newcommand{\baxiom}[1]{\btext{#1}}
\newcommand{\btype}[1]{\btext{#1}}
\newcommand{\bguard}[1]{\btext{#1}}
\newcommand{\bmachine}[1]{\btext{#1}}
\newcommand{\bctx}[1]{\btext{#1}}

\author{Matthias Güdemann\\Systerel, France}

\affiliation{Systerel}

\title{Event-B Model of Subset 026, Section 5.9}

% define the coverart
\coverart[width=350pt]{chart}

\reporttype{Model Description}

%\begin{document}

\maketitle
\tableofcontents
\listoffiguresandtables
\newpage

This document describes a formal model of the requirements of section~5.9 of
the subset 026 of the ETCS specification 3.3.0~\cite{SRS-026-330}. This section
describes the on-sight procedure.

The model is expressed in the formal language Event-B~\cite{abrial-eventB-Book}
and developed within the Rodin tool~\cite{rodin-handbook}. This formalism allows
an iterative modeling approach. In general, one starts with a very abstract
description of the basic functionality and step-wise adds additional details
until the desired level of accuracy of the model is reached. Rodin provides the
necessary proof support to ensure the correctness of the refined behavior.

In this document we present an Event-B model of the procedure on-sight. We use
the iUML plugin which allows for modeling in UML state-charts to create a
graphical model of the procedure which is as close as possible as its
description as flowchart in the section 5.9. The state machine is iteratively
developed using the refinement feature of Event-B. At each refinement step, we
present the reasoning for the step, together with newly introduced variables and
events.

\begin{table}[ht]
  \centering
  \begin{tabular}[ht]{|l|l|}
    \hline
    \hline
  \end{tabular}
  \caption{Glossary}
  \label{tab:glossary}
\end{table}

\section{Short Introduction to Event-B}
\label{sec:short-intr-event}

The formal language Event-B is based on a set-theoretic approach. It is a
variant of the B language, with a focus on system level
modeling~\cite{abrial-eventB-Book}. An Event-B model is separated into a static
and a dynamic part.

The dynamic part of an Event-B model describes abstract state machines. The
state is represented by a set of state variables. A transition from one state to
another is represented by parametrized events which assign new values to the
state variables. Event-B allows unbounded state spaces. They are constrained by
invariants expressed in first order logic with equality which must be fulfilled
in any case. The initial state is created by a special initialization event.

The static part of an Event-B model is represented by contexts. These consist of
carrier sets, constants and axioms. The type system of a model is described by
means of carrier sets and constraints expressed by axioms.

Event-B is not only comprised of descriptions of abstract state machines and
contexts, but also includes a development approach. This approach consists of
iterative refinement of the machines until the desired level of detail is
reached. In the Rodin tool, proof obligations are automatically created which
ensure correct refinement.

Together with the machine invariants, the proof obligations for the refinement
are formally proven, creating proof trees. To accomplish this, there are
different options: many proof obligations can be discharged by automated provers
(e.g., AtelierB, NewPP, Rodin's SMT-plugin), but as the underlying logic is in
general undecidable, it is sometimes necessary to use the interactive proof
support of Rodin.

Any external actions, e.g., mode changes by the driver or train level changes
are modeled via parametrized events. Only events can modify the variables of a
machine. An Event-B model is on the system level, events are assumed to be
called from a software system into which the functional model is embedded. The
guards of the events assure that any event can only be called when appropriate.

\section{Modeling Strategy}
\label{sec:modeling-strategy}

The section~5.9 of the SRS describes the procedure on-sight, in particular it
describes the sequence of mode changes, necessary driver acknowledge and train
brake to enter OS mode, dependent on the current train mode.

For better understanding and to automate many tasks for state based modeling, we
use the iUML plugin~\cite{UML-plugin} which automatically generates Event-B code
representing a state machine specification.

\section{Model Overview}
\label{sec:model-overview}

Figure~\ref{fig:model-overview} shows the structure of the Event-B model. The
left column represents the abstract state machines, the right column the
contexts. An arrow from one machine to another machine represents a refinement
relation, an arrow from a machine to a context represents a sees relation and
arrow from one context to another represents an extension relation.

\begin{figure}[ht]
  \centering
  \includegraphics[width=.3\textwidth]{SubSet_026_5_9}
  \caption{Overview on State Machine and Context Hierarchy}
  \label{fig:model-overview}
\end{figure}

The modeling starts with the very abstract possibility to establish and to
terminate a communication session in the machine \bmachine{m0}, the set of
entities is defined in the context \bctx{c0}. This basic functionality is
refined in the succeeding machines to incorporate a more detailed description of
the flowchart.

\section{Model Benefits}
\label{sec:model-highlights}

The Event-B model in Rodin has some interesting properties which are highlighted
here. Some stem from the fact that Rodin is well integrated into the Eclipse
platform which renders many useful plugins available, both those explicitly
developed for integration with Rodin, but also other without Rodin in mind.
Other interesting properties stem from the fact that Rodin and Event-B provide
an extensive proof support for properties.

\begin{itemize}
\item {\bf Graphical Modeling} Through the iUML plugin, Rodin supports graphical
  modeling of UML/SysML state machines. Transitions are labeled with events and
  a fully automatic transformation~\cite{said2009language} creates an Event-B
  representation of the state machine models.
\item {\bf Refinement} In addition to the general refinement which is possible
  in the Event-B approach, the graphical modeling allows to refine the graphical
  state chart models too. For each refinement step, the new details are
  graphically emphasized.
\item {\bf Model Animation} Through the ProB plugin, the graphical models can be
  animated just as textual Event-B models. In this case active transitions can
  be highlighted which helps understand model behavior.
\item {\bf Safety Properties} Using Rodin's proof support and the formalization
  as invariants, it is possible to formalize and prove the identified safety
  properties of the case study (see Section~\ref{sec:machine-3-accepting}).
\end{itemize}

\section{Detailed Model Description}
\label{sec:deta-model-descr}

This section describes in more detail the formal model, beginning from the most
abstract Event-B machine.  For each refinement, the state machine will be shown
and in general only the important manual changes in the model generated from the
state machine. The full generated code and the manual changes are available as a
Rodin project. At each step the additional modeled functionality and its
representation will be described. In particular the initialization event is not
shown for the refined machines. If not mentioned explicitly, sets are
initialized empty, integers with value 0 and Boolean variables with false.

\begin{figure}[ht]
  \centering
  \includegraphics[width=.75\textwidth]{FlowChart}
  \caption{Flowchart for "On-Sight" Procedure}
  \label{fig:flowchart-OS-mode}
\end{figure}

\subsection{Machine 0 - Basic Flowchart}
\label{sec:machine-0-basic}

This state machine (see Fig.~\ref{fig:basic-flowchart}) represents the basic
flowchart shown in §5.9.7 of the SRS~\cite{SRS-026-330} (see Fig.~




\begin{figure}[ht]
  \centering
  \includegraphics[width=.85\textwidth]{m0_basic_flowchart_on_sight_procedure}
  \caption{Basic Flowchart Representation}
  \label{fig:basic-flowchart}
\end{figure}


It allows for the creation and the termination of a communication session with
another entity. The sessions are represented by the state variable
\bvar{session} which can contain values of type \btype{entities}. The respective
events are triggered with a parameter \bparam{l\_partner} representing the
communication partner.


\subsection{Context 0 - Train Modes}
\label{sec:context-0-entities}

The first context \bvar{c0} specifies the possible modes of the train. There is
one Event-B constant for each possible mode. The constant
\bconst{c\_initial\_mode} represents the initial mode of the train when the
procedure on-sight is started. The constant \bconst{c\_supervision\_mode} is one
mode from the supervision modes.

\begin{description}
\SETS
	\begin{description}
		\Item{ t\_train\_modes }
	\end{description}
\CONSTANTS
	\begin{description}
		\ItemY{ c\_FS }{full supervision}
		\ItemY{ c\_LS }{limited supervision}
		\ItemY{ c\_OS }{on sight}
		\ItemY{ c\_SR }{staff responsible}
		\ItemY{ c\_SB }{stand-by}
		\ItemY{ c\_PT }{post-trip}
		\ItemY{ c\_UN }{unfitted}
		\ItemY{ c\_SN }{national system}
		\Item{ c\_initial\_mode }
		\Item{ c\_supervision\_mode }
	\end{description}
\AXIOMS
	\begin{description}
		\nItem{ axm1 }{ partition(t\_train\_modes,\{ c\_FS\} ,\{ c\_LS\} ,\{ c\_OS\} ,\{ c\_SR\} ,		\\\hspace*{6 cm}  \{ c\_SB\} ,\{ c\_PT\} ,\{ c\_UN\} ,\{ c\_SN\} ) }		\nItem{ axm2 }{ c\_initial\_mode \in  \{ c\_FS, c\_OS, c\_PT\}  }		\nItem{ axm3 }{ c\_supervision\_mode \in  \{ c\_LS, c\_FS\}  }	\end{description}
\END
\end{description}



\subsection{Machine 1 - Directional Communication}
\label{sec:mach-1-direct}

The first refinement of the machine refines the notion of communication session
to incoming sessions, i.e., where another entity initiates a session with
\bvar{my\_entity} and outgoing sessions where \bvar{my\_entity} initiates the
session.

The data refinement is proven by the invariant which states that \bvar{sessions}
is equal to the disjoint union of \bvar{outgoing\_sessions} and
\bvar{incoming\_sessions}. The abstract \bvar{establish\_session} event is
refined to the two events \bvar{incoming\_session} and \bvar{outgoing\_session}.

%\begin{description}
\VARIABLES
        \begin{description}
                \Item{ incoming\_sessions }
                \Item{ outgoing\_sessions }
        \end{description}
\INVARIANTS
        \begin{description}
                \nItemY{ inv1 }{ comm\_sessions = incoming\_sessions \bunion  outgoing\_sessions }{  }
                \nItem{ inv2 }{ incoming\_sessions \binter  outgoing\_sessions = \emptyset  }
        \end{description}
\EVENTS
        \EVT {incoming\_communication}
        \REF {establish\_communication}
                \begin{description}
                \AnyPrm
                        \begin{description}
                        \ItemY{l\_partner }{ }
                        \end{description}
                \WhereGrd
                        \begin{description}
                        \nItem{ grd1 }{ l\_partner \notin  incoming\_sessions \bunion  outgoing\_sessions }
                        \nItemY{ grd2 }{ l\_partner \in  entities \setminus  \{ l\_partner\}  }{ }
                        \end{description}
                \ThenAct
                        \begin{description}
                        \nItemY{ act1 }{ incoming\_sessions :=  incoming\_sessions \bunion  \{ l\_partner\}  }{  }
                        \end{description}
                \EndAct
                \end{description}
        \EVT {outgoing\_communciation}
        \REF {establish\_communication}
                \begin{description}
                \AnyPrm
                        \begin{description}
                        \ItemY{l\_partner }{ }
                        \end{description}
                \WhereGrd
                        \begin{description}
                        \nItem{ grd2 }{ l\_partner \in  entities \setminus  \{ l\_partner\}  }
                        \nItemY{ grd1 }{ l\_partner \notin  incoming\_sessions \bunion  outgoing\_sessions }{ }
                        \end{description}
                \ThenAct
                        \begin{description}
                        \nItem{ act1 }{ outgoing\_sessions :=  outgoing\_sessions \bunion  \{ l\_partner\}  }
                        \end{description}
                \EndAct
                \end{description}
\end{description}


\subsection{Context 1 - Entity Types}
\label{sec:context-1-entity}

The first context extension introduces the different types of entities relevant
in this requirement subset, i.e., on-board unit (OBU), radio in-fill unit (RIU)
or radio block centre (RBC). Every entity has a unique type. The goal is to
model the communication protocol from the point of view of an OBU, therefore the
type of \bvar{my\_entity} is restricted to OBU.

%\begin{description}
\CONTEXT{c1\_entity\_types}
\EXTENDS{c0\_entities}
\CONSTANTS
        \begin{description}
                \Item{ RBC }
                \Item{ RIU }
                \ItemY{ OBU }{}
                \Item{ on\_track }
                \ItemY{ on\_board }{}
        \end{description}
\AXIOMS
        \begin{description}
                \nItem{ axm1 }{ partition(entities,RBC,RIU,OBU) }
                \nItem{ axm2 }{ on\_track = RIU \bunion  RBC }
                \nItem{ axm3 }{ on\_board = OBU }
                \nItem{ axm4 }{ my\_entity \in  on\_board }
              \end{description}
\END
\end{description}


\subsection{Machine 2 - On Board Modeling}
\label{sec:machine-2-board}

The next machine refinement adds the notion of being contacted by an on-track
entity to establish a communication session. It also adds the first state of the
protocol, i.e., entities which should be contacted with the ``Initiation of a
communication session'' message. On-track entities which order \bvar{my\_entity}
to contact are stored in the \bvar{contacted\_by} set, entities to which the
first message is sent by \bvar{my\_entity} are stored in the set
\bvar{contacted}, representing those which are in the first stage of the
protocol.

The invariants prove that \bvar{my\_entity} will only be in contact with
on-track entities and that any entities which are considered for a communication
session are on-track entities. Any entity with whom there is already a
communication session will not be considered for another session, and finally no
radio in-fill unit can initiate a communication session with \bvar{my\_entity}.

\paragraph{Implemented Requirements}
\label{sec:impl-requ-1}

\begin{itemize}
\item It shall be possible for OBU and RBC to initiate communication session
  (cf.~§3.5.3.1)
\item RIU cannot initiate a communication session (cf.~§3.5.3.2)

This invariant is marked as non-testable in Subset-076.
\end{itemize}

The other invariants ensure that a communication partner is not in different
states of the communication protocol at the same time. A session protocol can be
started by the order to contact an RBC or directly by the OBU.

%\begin{description}
\REFINES{m1\_directional\_communication}
\SEES{c1\_entity\_types}
\VARIABLES
        \begin{description}
                \Item{ contacted }
                \ItemY{ contacted\_by }{}
        \end{description}
\INVARIANTS
        \begin{description}
                \nItemY{ inv1 }{ incoming\_sessions \bunion  outgoing\_sessions \subseteq  on\_track }{  }
                \nItemY{ inv2 }{ contacted \subseteq  on\_track }{  }
                \nItemY{ inv3 }{ contacted\_by \subseteq  on\_track }{  }
                \nItem{ inv4 }{ contacted\_by \binter  (incoming\_sessions \bunion  outgoing\_sessions) = \emptyset  }
                \nItemY{ inv5 }{ contacted \binter  (incoming\_sessions \bunion  outgoing\_sessions) = \emptyset  }{  }
                \nItemY{ inv6 }{ incoming\_sessions \binter  RIU = \emptyset  }{  }
                \nItem{ inv7 }{ contacted \binter  contacted\_by = \emptyset  }
        \end{description}
\EVENTS
        \EVT {incoming\_communication}
        \REF {incoming\_communication}
                \begin{description}
                \AnyPrm
                        \begin{description}
                        \ItemY{l\_partner }{ }
                        \end{description}
                \WhereGrd
                        \begin{description}
                        \nItem{ grd1 }{ l\_partner \notin  incoming\_sessions \bunion  outgoing\_sessions }
                        \nItemY{ grd3 }{ l\_partner \in  on\_track \setminus  RIU }{ }
                        \nItemY{ grd4 }{ l\_partner \notin  contacted }{ }
                        \nItem{ grd5 }{ l\_partner \notin  contacted\_by }
                        \end{description}
                \ThenAct
                        \begin{description}
                        \nItemY{ act1 }{ incoming\_sessions :=  incoming\_sessions \bunion  \{ l\_partner\}  }{  }
                        \end{description}
                \EndAct
                \end{description}
        \EVT {receive\_contact\_order}\cmt{}
                \begin{description}
                \AnyPrm
                        \begin{description}
                        \Item{l\_partner }
                        \end{description}
                \WhereGrd
                        \begin{description}
                        \nItemY{ grd1 }{ l\_partner \notin  contacted \bunion  contacted\_by \bunion  incoming\_sessions \bunion  outgoing\_sessions }{ }
                        \nItem{ grd2 }{ l\_partner \in  on\_track }
                        \end{description}
                \ThenAct
                        \begin{description}
                        \nItem{ act1 }{ contacted\_by :=  contacted\_by \bunion  \{ l\_partner\}  }
                        \end{description}
                \EndAct
                \end{description}
        \EVT {initiate\_session\_after\_contact}\cmt{}
                \begin{description}
                \AnyPrm
                        \begin{description}
                        \Item{l\_partner }
                        \end{description}
                \WhereGrd
                        \begin{description}
                        \nItemY{ grd2 }{ l\_partner \in  contacted\_by }{ }
                        \end{description}
                \ThenAct
                        \begin{description}
                        \nItemY{ act1 }{ contacted :=  contacted \bunion  \{ l\_partner\}  }{  }
                        \nItemY{ act2 }{ contacted\_by :=  contacted\_by \setminus  \{ l\_partner\}  }{  }
                        \end{description}
                \EndAct
                \end{description}
        \EVT {initiate\_session\_no\_contact}\cmt{ }
                \begin{description}
                \AnyPrm
                        \begin{description}
                        \ItemY{l\_partner }{ }
                        \end{description}
                \WhereGrd
                        \begin{description}
                        \nItem{ grd5 }{ l\_partner \notin  incoming\_sessions \bunion  outgoing\_sessions \bunion  contacted \bunion  contacted\_by }
                        \nItemY{ grd3 }{ l\_partner \in  on\_track }{ }
                        \end{description}
                \ThenAct
                        \begin{description}
                        \nItemY{ act2 }{ contacted :=  contacted \bunion  \{ l\_partner\}  }{  }
                        \end{description}
                \EndAct
                \end{description}
\END
\end{description}


\subsection{Context 2 - System Version Compatibility}
\label{sec:context-2-system}

The next context extension introduces the notion of compatible system
versions. This is modeled as a static property of the on-track equipment
wrt. \bvar{my\_entity}, i.e., the context axiom that
\bvar{system\_version\_compatible} is a subset of all on-track entities. On this
level of abstraction, there is no need for a finer grained notion of
compatibility.

%% \documentclass[10pt,a4paper]{report}
% \usepackage[top=3cm, bottom=2.5cm, left=3cm, right=2.5cm] {geometry}
% \usepackage {bsymb,b2latex}
% \usepackage[utf8]{inputenc}
% \usepackage{fancyhdr,lastpage,color}
% \lhead{\rm An Event-B Specification of c2\_system\_version\_mode}
% \rhead {\rm Page \thepage~of \pageref{LastPage}}
% \lfoot{}\cfoot{}\rfoot{}
% \pagestyle{fancy}
% %---------------------------------------------------------
% \begin{document}
% \thispagestyle{empty}
\begin{description}
\EXTENDS{c1\_entity\_types}
\CONSTANTS
        \begin{description}
                \Item{ system\_version\_compatible }
        \end{description}
\AXIOMS
        \begin{description}
                \nItem{ axm1 }{ system\_version\_compatible \subseteq  on\_track }	\end{description}
\END
\end{description}


\subsection{Machine 3 - Accepting RBC and System Version}
\label{sec:machine-3-accepting}

The next machine refines the contact order events by discerning between the
orders to contact an accepting or a non-accepting RBC. The notion of being an
accepting RBC is considered to be a dynamic property and therefore modeled as a
variable, i.e., the set \bvar{accepting}.

The \bvar{receive\_contact\_order} event is refined by two separate events, one
for orders for an accepting RBC and one for a non-accepting RBC. The
\bvar{outgoing\_communication} event is refined by two events, one for a
compatible system version and the other for an incompatible one. The events for
\bevent{initiate\_session\_contact} and \bevent{initiate\_session\_no\_contact}
are analogously refined to two version for accepting and non-accepting RBCs.

Furthermore, a just established communication session with on-track equipment
with an incompatible system version will be terminated immediately after
receiving this information. This is modeled by the set
\bvar{terminating\_session} which holds values of type entities. Only those
communication sessions in this set can be closed by the termination event.

\paragraph{Implemented Requirements}
\label{sec:impl-requ-2}

\begin{itemize}
\item In case of a non-accepting RBC, all existing communication sessions with
  other RBCs must be terminated (cf.~§3.5.3.5.2)
\item After the system version  is received by the OBU, the communication
  session is considered established and (cf.~§3.5.3.8)
  \begin{itemize}
  \item if the system version is compatible, the OBU shall send the session
    established message to track-side (cf.~3.5.3.8.a)
  \item if the system version is incompatible, the OBU shall terminate the
    session (cf.~3.5.3.8.b)
  \end{itemize}
\item Any RBC which is contacted and with whom a communication session is
  established has a compatible system version (safety requirement from
  requirements document).
\end{itemize}

%\begin{description}
\REFINES{m2\_limit\_OBU}
\SEES{c2\_system\_version\_mode}
\VARIABLES
        \begin{description}
                \ItemY{ termination\_sessions }{}
                \Item{ accepting }
        \end{description}
\INVARIANTS
        \begin{description}
                \nItemY{ inv1 }{ termination\_sessions \subseteq  incoming\_sessions \bunion  outgoing\_sessions }{             \\\hspace*{1,4 cm}  only established sessions can be terminated }
                \nItemY{ inv2 }{ RBC \binter  outgoing\_sessions \subseteq  system\_version\_compatible }{              \\\hspace*{1,4 cm}  Sylvain's safety property }
                \nItemY{ inv3 }{ accepting \subseteq  RBC }{            \\\hspace*{1,4 cm}  typing invariant for elements of accepting }
        \end{description}
\EVENTS
        \EVT {receive\_information\_compatible}
        \EXTD {outgoing\_communciation}
                \begin{description}
                \AnyPrm
                        \begin{description}
                        \ItemXY{l\_partner }{ }
                        \end{description}
                \WhereGrd
                        \begin{description}
                        \nItemX{ grd2 }{ l\_partner \in  entities \setminus  \{ l\_partner\}  }
                        \nItemXY{ grd1 }{ l\_partner \notin  incoming\_sessions \bunion  outgoing\_sessions }{ }
                        \nItemX{ grd3 }{ l\_partner \in  on\_track }
                        \nItem{ grd4 }{ l\_partner \in  system\_version\_compatible }
                        \end{description}
                \ThenAct
                        \begin{description}
                        \nItemX{ act1 }{ outgoing\_sessions :=  outgoing\_sessions \bunion  \{ l\_partner\}  }
                        \end{description}
                \EndAct
                \end{description}
        \EVT {receive\_information\_incompatible}
        \EXTD {outgoing\_communciation}
                \begin{description}
                \AnyPrm
                        \begin{description}
                        \ItemXY{l\_partner }{ }
                        \end{description}
                \WhereGrd
                        \begin{description}
                        \nItemX{ grd2 }{ l\_partner \in  entities \setminus  \{ l\_partner\}  }
                        \nItemXY{ grd1 }{ l\_partner \notin  incoming\_sessions \bunion  outgoing\_sessions }{ }
                        \nItemX{ grd3 }{ l\_partner \in  on\_track }
                        \nItem{ grd4 }{ l\_partner \notin  system\_version\_compatible }
                        \end{description}
                \ThenAct
                        \begin{description}
                        \nItemX{ act1 }{ outgoing\_sessions :=  outgoing\_sessions \bunion  \{ l\_partner\}  }
                        \nItem{ act2 }{ termination\_sessions :=  termination\_sessions \bunion  \{ l\_partner\}  }
                        \end{description}
                \EndAct
                \end{description}
        \EVT {receive\_contact\_order\_accept}\cmt{		\\\hspace*{6,2 cm}  order to contact a RIU or accepting RBC }
        \EXTD {receive\_contact\_order}
                \begin{description}
                \AnyPrm
                        \begin{description}
                        \ItemX{l\_partner }
                        \end{description}
                \WhereGrd
                        \begin{description}
                        \nItemXY{ grd1 }{ l\_partner \notin  contacted \bunion  contacted\_by \bunion  incoming\_sessions \bunion  outgoing\_sessions }{ }
                        \nItemX{ grd2 }{ l\_partner \in  on\_track }
                        \nItemY{ grd3 }{ l\_partner \in  RIU \bunion  (RBC \binter  accepting) }{		\\\hspace*{1,4 cm}  either RIU or accepting RBC }
                        \end{description}
                \ThenAct
                        \begin{description}
                        \nItemX{ act1 }{ contacted\_by :=  contacted\_by \bunion  \{ l\_partner\}  }
                        \end{description}
                \EndAct
                \end{description}
        \EVT {receive\_contact\_order\_non\_accept}\cmt{		\\\hspace*{7 cm}  trackside can order contact (cf. 3.5.3.4.b) }
        \EXTD {receive\_contact\_order}
                \begin{description}
                \AnyPrm
                        \begin{description}
                        \ItemX{l\_partner }
                        \end{description}
                \WhereGrd
                        \begin{description}
                        \nItemXY{ grd1 }{ l\_partner \notin  contacted \bunion  contacted\_by \bunion  incoming\_sessions \bunion  outgoing\_sessions }{ }
                        \nItemX{ grd2 }{ l\_partner \in  on\_track }
                        \nItem{ grd3 }{ l\_partner \in  RIU \bunion  (RBC \setminus  (RBC \binter  accepting)) }
                        \end{description}
                \ThenAct
                        \begin{description}
                        \nItemX{ act1 }{ contacted\_by :=  contacted\_by \bunion  \{ l\_partner\}  }
                        \nItem{ act2 }{ termination\_sessions :=  termination\_sessions \bunion  (RBC \binter  (incoming\_sessions \bunion  outgoing\_sessions)) }
                        \end{description}
                \EndAct
                \end{description}
        \EVT {terminate\_communication}
        \EXTD {terminate\_communication}
                \begin{description}
                \AnyPrm
                        \begin{description}
                        \ItemX{l\_partner }
                        \end{description}
                \WhereGrd
                        \begin{description}
                        \nItemX{ grd1 }{ l\_partner \in  incoming\_sessions \bunion  outgoing\_sessions }
                        \nItem{ grd2 }{ l\_partner \in  termination\_sessions }
                        \end{description}
                \ThenAct
                        \begin{description}
                        \nItemX{ act1 }{ incoming\_sessions :=  incoming\_sessions \setminus  \{ l\_partner\}  }
                        \nItemX{ act2 }{ outgoing\_sessions :=  outgoing\_sessions \setminus  \{ l\_partner\}  }
                        \nItem{ act3 }{ termination\_sessions :=  termination\_sessions \setminus  \{ l\_partner\}  }
                        \end{description}
                \EndAct
                \end{description}
        \EVT {make\_RBC\_accepting}
                \begin{description}
                \AnyPrm
                        \begin{description}
                        \ItemY{l\_partner }{ }
                        \end{description}
                \WhereGrd
                        \begin{description}
                        \nItem{ grd1 }{ l\_partner \in  RBC }
                        \end{description}
                \ThenAct
                        \begin{description}
                        \nItem{ act1 }{ accepting :=  accepting \bunion  \{ l\_partner\}  }
                        \end{description}
                \EndAct
                \end{description}
        \EVT {make\_RBC\_non\_accepting}
                \begin{description}
                \AnyPrm
                        \begin{description}
                        \Item{l\_partner }
                        \end{description}
                \WhereGrd
                        \begin{description}
                        \nItem{ grd1 }{ l\_partner \in  accepting }
                        \end{description}
                \ThenAct
                        \begin{description}
                        \nItem{ act1 }{ accepting :=  accepting \setminus  \{ l\_partner\}  }
                        \end{description}
                \EndAct
                \end{description}
\END
\end{description}


\subsection{Context 3 - ERTMS Levels}
\label{sec:context-3-ertms}

The third context introduces the notion of the different ERTMS and the notion of
the mission status of a train. The modeled statuses are start of mission (SOM),
end of mission (EOM) and the abstract notion of within a mission (MIS), i.e.,
anything between start and end of the current train mission. At this level of
refinement, a more detailed modeling is not necessary.

%\begin{description}
\CONTEXT{c3\_ERTMS\_level}
\SETS
        \begin{description}
                \ItemY{ ERTMS\_level }{ }
                \Item{ train\_status }
        \end{description}
\CONSTANTS
        \begin{description}
                \ItemY{ NTC }{}
                \Item{ L0 }
                \Item{ L1 }
                \Item{ L2 }
                \ItemY{ L3 }{}
                \ItemY{ SOM }{start of mission}
                \ItemY{ EOM }{end of mission}
                \ItemY{ MIS }{while mission}
        \end{description}
\AXIOMS
        \begin{description}
                \nItem{ axm1 }{ partition(ERTMS\_level, \{ NTC, L0, L1, L2, L3\} ) }		\nItem{ axm2 }{ partition(train\_status, \{ SOM, EOM, MIS\} ) }	\end{description}
\END
\end{description}



\subsection{Machine 4 - ERTMS Level Changes}
\label{sec:machine-4-ertms}

The next refined machine implements the different causes which can trigger the
establishing of a communication session. The corresponding events refine the
abstract \bvar{initiate\_session\_no\_contact} events. For this, the current
ERTMS level of the train is tracked, as well as its current mission status.

The indication of a level change, a mission status change, a manual level change
and an announced radio hole is modeled by events. These events modify the
corresponding indicator variables to signal a change and they modify the
corresponding state variables.

This can be illustrated using the manual level change event as example: the
Boolean variable \bvar{signal\_manual\_level\_change} indicates that the driver
manually changed the ERTMS level. It is changed by the
\bvar{manual\_change\_level} event which is parametrized with the new level and
which also modifies the \bvar{current\_level} variable which models the current
ERTMS level. If the new level is 2 or 3, then the train is required to establish
a communication session with trackside. This is realized in the
\bvar{initiate\_session\_no\_contact\_manual\_change} events which reset the
indication variable once the entity has been contacted.  Similar events model
the initiation because of non-manual level change, mission status change and
announced radio holes, each in a version for accepting and non-accepting RBCs.

\paragraph{Implemented Requirements}
\label{sec:impl-requ-3}

\begin{itemize}
\item The on-board shall establish a communication session (cf.~§3.5.3.4)
  \begin{itemize}
  \item at start of mission (only if level 2 or 3) (cf.~§3.5.3.4.a)
  \item if ordered from trackside (cf.~§3.5.3.4.b)
  \item If a mode change, not considered as an End of Mission, has to be
    reported to the RBC (only if level 2 or 3) (cf.~§3.5.3.4.c)
  \item If the driver has manually changed the level to 2 or 3 (cf.~§3.5.3.4.d)
  \item When the train front reaches the end of an announced radio hole
    (cf.~§3.5.3.4.e)
  \end{itemize}
\end{itemize}

%\begin{description}
\REFINES{m3\_accept\_system\_version}
\SEES{c3\_ERTMS\_level}
\VARIABLES
        \begin{description}
                \ItemY{ current\_level }{}
                \ItemY{ signal\_level\_change }{}
                \ItemY{ current\_status }{}
                \ItemY{ signal\_status\_change }{}
                \ItemY{ signal\_manual\_level\_change }{}
                \ItemY{ position\_radio\_hole }{}
                \Item{ signal\_radio\_hole }
        \end{description}
\INVARIANTS
        \begin{description}
                \nItemY{ inv1 }{ current\_level \in  ERTMS\_level }{  }
                \nItemY{ inv2 }{ signal\_level\_change \in  BOOL }{  }
                \nItem{ inv3 }{ current\_status \in  train\_status }
                \nItemY{ inv4 }{ signal\_status\_change \in  BOOL }{  }
                \nItemY{ inv5 }{ signal\_manual\_level\_change \in  BOOL }{  }
                \nItemY{ inv6 }{ position\_radio\_hole \in  BOOL }{  }
                \nItem{ inv7 }{ signal\_radio\_hole \in  BOOL }
        \end{description}
\EVENTS
        \EVT {manual\_change\_level}
                \begin{description}
                \AnyPrm
                        \begin{description}
                        \ItemY{l\_level }{ }
                        \end{description}
                \WhereGrd
                        \begin{description}
                        \nItemY{ grd1 }{ l\_level \in  ERTMS\_level }{ }
                        \nItem{ grd2 }{ signal\_manual\_level\_change = FALSE }
                        \nItem{ grd3 }{ signal\_level\_change = FALSE }
                        \end{description}
                \ThenAct
                        \begin{description}
                        \nItemY{ act1 }{ signal\_manual\_level\_change :=  TRUE }{  }
                        \nItem{ act2 }{ current\_level :=  l\_level }
                        \end{description}
                \EndAct
                \end{description}
        \EVT {change\_level}
                \begin{description}
                \AnyPrm
                        \begin{description}
                        \ItemY{l\_level }{ }
                        \end{description}
                \WhereGrd
                        \begin{description}
                        \nItemY{ grd1 }{ l\_level \in  ERTMS\_level }{ }
                        \nItemY{ grd2 }{ signal\_manual\_level\_change = FALSE }{ }
                        \nItem{ grd3 }{ signal\_level\_change = FALSE }
                        \end{description}
                \ThenAct
                        \begin{description}
                        \nItemY{ act1 }{ current\_level :=  l\_level }{  }
                        \nItem{ act2 }{ signal\_level\_change :=  TRUE }
                        \end{description}
                \EndAct
                \end{description}
        \EVT {initiate\_session\_no\_contact\_accept}
        \EXTD {initiate\_session\_no\_contact\_accept}
                \begin{description}
                \AnyPrm
                        \begin{description}
                        \ItemXY{l\_partner }{ }
                        \end{description}
                \WhereGrd
                        \begin{description}
                        \nItemX{ grd5 }{ l\_partner \notin  incoming\_sessions \bunion  outgoing\_sessions \bunion  contacted \bunion  contacted\_by }
                        \nItemXY{ grd3 }{ l\_partner \in  RIU \bunion  (RBC \binter  accepting) }{ }
                        \end{description}
                \ThenAct
                        \begin{description}
                        \nItemXY{ act2 }{ contacted :=  contacted \bunion  \{ l\_partner\}  }{  }
                        \end{description}
                \EndAct
                \end{description}
        \EVT {initiate\_session\_no\_contact\_non\_accept}
        \EXTD {initiate\_session\_no\_contact\_non\_accept}
                \begin{description}
                \AnyPrm
                        \begin{description}
                        \ItemXY{l\_partner }{ }
                        \end{description}
                \WhereGrd
                        \begin{description}
                        \nItemX{ grd5 }{ l\_partner \notin  incoming\_sessions \bunion  outgoing\_sessions \bunion  contacted \bunion  contacted\_by }
                        \nItemXY{ grd3 }{ l\_partner \in  RIU \bunion  (RBC \setminus  accepting) }{ }
                        \end{description}
                \ThenAct
                        \begin{description}
                        \nItemXY{ act2 }{ contacted :=  contacted \bunion  \{ l\_partner\}  }{  }
                        \nItemX{ act3 }{ terminating\_sessions :=  terminating\_sessions \bunion  (RBC \binter  (incoming\_sessions \bunion  outgoing\_sessions)) }
                        \end{description}
                \EndAct
                \end{description}
        \EVT {initiate\_session\_no\_contact\_manual\_change\_accept}
        \EXTD {initiate\_session\_no\_contact\_accept}
                \begin{description}
                \AnyPrm
                        \begin{description}
                        \ItemXY{l\_partner }{ }
                        \end{description}
                \WhereGrd
                        \begin{description}
                        \nItemX{ grd5 }{ l\_partner \notin  incoming\_sessions \bunion  outgoing\_sessions \bunion  contacted \bunion  contacted\_by }
                        \nItemXY{ grd3 }{ l\_partner \in  RIU \bunion  (RBC \binter  accepting) }{ }
                        \nItem{ grd6 }{ current\_level \in  \{ L2, L3\}  }
                        \nItem{ grd7 }{ signal\_manual\_level\_change = TRUE }
                        \end{description}
                \ThenAct
                        \begin{description}
                        \nItemXY{ act2 }{ contacted :=  contacted \bunion  \{ l\_partner\}  }{  }
                        \nItem{ act3 }{ signal\_manual\_level\_change :=  FALSE }
                        \end{description}
                \EndAct
                \end{description}
        \EVT {initiate\_session\_no\_contact\_manual\_change\_non\_accept}
        \REF {initiate\_session\_no\_contact\_non\_accept}
                \begin{description}
                \AnyPrm
                        \begin{description}
                        \ItemY{l\_partner }{ }
                        \end{description}
                \WhereGrd
                        \begin{description}
                        \nItem{ grd5 }{ l\_partner \notin  incoming\_sessions \bunion  outgoing\_sessions \bunion  contacted \bunion  contacted\_by }
                        \nItemY{ grd3 }{ l\_partner \in  RIU \bunion  (RBC \setminus  accepting) }{ }
                        \nItem{ grd6 }{ current\_level \in  \{ L2, L3\}  }
                        \nItem{ grd7 }{ signal\_manual\_level\_change = TRUE }
                        \end{description}
                \ThenAct
                        \begin{description}
                        \nItemY{ act2 }{ contacted :=  contacted \bunion  \{ l\_partner\}  }{  }
                        \nItem{ act3 }{ terminating\_sessions :=  terminating\_sessions \bunion  (RBC \binter  (incoming\_sessions \bunion  outgoing\_sessions)) }
                        \end{description}
                \EndAct
                \end{description}
\END
\end{description}


\subsection{Machine 5 - Safe Radio Connection}
\label{sec:machine-5-safe}

The next machine refinement specifies handling of the safe radio connection
which provides the necessary means to exchange protocol messages on a higher
level. The existing established safe radio connections are represented by the
set \bvar{ER\_connections} which holds values of type entities. Safe radio
connections which must be established are modeled by the set
\bvar{establish\_ER\_connections} while \bvar{terminated\_ER\_connections} holds
those connections which timed-out. The indication variable
\bvar{signal\_RBC\_border} signals the crossing of an RBC border which requires
to establish a new safe radio connection with a new RBC.

Establishing a communication session then works as follows: if one of the
conditions of §3.5.3.4 is fulfilled, the corresponding partner is added to the
set \bvar{contacted}. This is a pre-condition of the events which open a safe
radio connection to a communication partner. The initiation message of the
protocol is considered to be sent when a communication partner is both
\bvar{contacted} and \bvar{ER\_connections} set. The reception of the system
version and the sending of the system established message is modeled via the
\bvar{receive\_information\_compatible} or
\bvar{receive\_information\_incompatible} events. Established sessions with
incompatible system versions are therefore in the intersection of the
\bvar{sessions} and \bvar{terminating\_sessions} sets.

\paragraph{Implemented Requirements}
\label{sec:impl-requ-4}

\begin{itemize}
\item Establish communication session after safe radio connection timeout
  (cf.~§3.5.3.4.f)
\item If the communication session is established by an OBU, it shall be
  preformed according to the following steps (cf.~§3.5.3.7)
  \begin{itemize}
  \item if part of ongoing start of mission procedure (cf.~§3.5.3.7.a)
  \item if safe radio connection is set up (cf.~§3.5.3.7.i)
  \item if order to terminate is received (cf.~§3.5.3.7.ii)
  \item if end of mission is performed (cf.~§3.5.3.7.iii)
  \item train passes level transition border (cf.~§3.5.3.7.iv)
  \item order to establish connection with different non-accepting RBC
    (cf.~§3.5.3.7.v)
  \item train passes RBC / RBC border (cf.~§3.5.3.7.vi)
  \item train enters announced radio hole (cf.~§3.5.3.7.vii)
  \item level 1 is left (RIU only) (cf.~§3.5.3.7.viii)
  \end{itemize}
\item When the RBC initiates the establishing of a communication session, the
  OBU shall consider the session established after receiving the initiation
  message (cf.~§3.5.3.10.c)
\end{itemize}

%\begin{description}
\REFINES{m4\_level\_changes}
\SEES{c3\_ERTMS\_level}
\VARIABLES
        \begin{description}
                \ItemY{ ER\_connections }{}
                \ItemY{ terminated\_ER\_connections }{}
                \ItemY{ establish\_ER\_connection }{}
                \Item{ signal\_RBC\_border }{}
        \end{description}
\INVARIANTS
        \begin{description}
                \nItemY{ inv1 }{ terminated\_ER\_connections \subseteq  on\_track }{  }
                \nItemY{ inv2 }{ establish\_ER\_connection \subseteq  on\_track }{  }
                \nItemY{ inv3 }{ (incoming\_sessions \bunion  outgoing\_sessions) \subseteq  ER\_connections }{  }
                \nItem{ inv4 }{ signal\_RBC\_border \in  BOOL }
        \end{description}
\EVENTS
        \EVT {incoming\_communication}
        \EXTD {incoming\_communication}
                \begin{description}
                \AnyPrm
                        \begin{description}
                        \ItemXY{l\_partner }{ }
                        \end{description}
                \WhereGrd
                        \begin{description}
                        \nItemX{ grd1 }{ l\_partner \notin  incoming\_sessions \bunion  outgoing\_sessions }
                        \nItemXY{ grd3 }{ l\_partner \in  on\_track \setminus  RIU }{ }
                        \nItemXY{ grd4 }{ l\_partner \notin  contacted }{ }
                        \nItemX{ grd5 }{ l\_partner \notin  contacted\_by }
                        \nItem{ grd6 }{ l\_partner \in  ER\_connections }
                        \end{description}
                \ThenAct
                        \begin{description}
                        \nItemXY{ act1 }{ incoming\_sessions :=  incoming\_sessions \bunion  \{ l\_partner\}  }{  }
                        \end{description}
                \EndAct
                \end{description}
        \EVT {receive\_information\_compatible}
        \EXTD {receive\_information\_compatible}
                \begin{description}
                \AnyPrm
                        \begin{description}
                        \ItemXY{l\_partner }{ }
                        \end{description}
                \WhereGrd
                        \begin{description}
                        \nItemXY{ grd3 }{ l\_partner \in  contacted }{ }
                        \nItemX{ grd4 }{ l\_partner \in  system\_version\_compatible }
                        \nItem{ grd5 }{ l\_partner \in  ER\_connections }
                        \end{description}
                \ThenAct
                        \begin{description}
                        \nItemX{ act1 }{ outgoing\_sessions :=  outgoing\_sessions \bunion  \{ l\_partner\}  }
                        \nItemX{ act2 }{ contacted :=  contacted \setminus  \{ l\_partner\}  }
                        \end{description}
                \EndAct
                \end{description}
        \EVT {receive\_information\_incompatible}
        \EXTD {receive\_information\_incompatible}
                \begin{description}
                \AnyPrm
                        \begin{description}
                        \ItemXY{l\_partner }{ }
                        \end{description}
                \WhereGrd
                        \begin{description}
                        \nItemXY{ grd3 }{ l\_partner \in  contacted }{ }
                        \nItemX{ grd4 }{ l\_partner \notin  system\_version\_compatible }
                        \nItem{ grd5 }{ l\_partner \in  ER\_connections }
                        \end{description}
                \ThenAct
                        \begin{description}
                        \nItemX{ act1 }{ outgoing\_sessions :=  outgoing\_sessions \bunion  \{ l\_partner\}  }
                        \nItemX{ act2 }{ contacted :=  contacted \setminus  \{ l\_partner\}  }
                        \nItemX{ act3 }{ terminating\_sessions :=  terminating\_sessions \bunion  \{ l\_partner\}  }
                        \end{description}
                \EndAct
                \end{description}
        \EVT {receive\_contact\_order\_accept}\cmt{		\\\hspace*{6,2 cm}  order to contact a RIU or accepting RBC }
        \EXTD {receive\_contact\_order\_accept}
                \begin{description}
                \AnyPrm
                        \begin{description}
                        \ItemX{l\_partner }
                        \end{description}
                \WhereGrd
                        \begin{description}
                        \nItemXY{ grd1 }{ l\_partner \notin  contacted \bunion  contacted\_by \bunion  incoming\_sessions \bunion  outgoing\_sessions }{ }
                        \nItemXY{ grd3 }{ l\_partner \in  RIU \bunion  (RBC \binter  accepting) }{}
                        \nItem{ grd4 }{ l\_partner \notin  terminated\_ER\_connections }
                        \end{description}
                \ThenAct
                        \begin{description}
                        \nItemX{ act1 }{ contacted\_by :=  contacted\_by \bunion  \{ l\_partner\}  }
                        \end{description}
                \EndAct
                \end{description}
        \EVT {receive\_contact\_order\_non\_accept}\cmt{}
        \EXTD {receive\_contact\_order\_non\_accept}
                \begin{description}
                \AnyPrm
                        \begin{description}
                        \ItemX{l\_partner }
                        \end{description}
                \WhereGrd
                        \begin{description}
                        \nItemXY{ grd1 }{ l\_partner \notin  contacted \bunion  contacted\_by \bunion  incoming\_sessions \bunion  outgoing\_sessions }{ }
                        \nItemX{ grd3 }{ l\_partner \in  RIU \bunion  (RBC \setminus  accepting) }
                        \nItem{ grd4 }{ l\_partner \notin  terminated\_ER\_connections }
                        \end{description}
                \ThenAct
                        \begin{description}
                        \nItemX{ act1 }{ contacted\_by :=  contacted\_by \bunion  \{ l\_partner\}  }
                        \nItemX{ act2 }{ terminating\_sessions :=  terminating\_sessions \bunion  (RBC \binter  (incoming\_sessions \bunion  outgoing\_sessions)) }
                        \end{description}
                \EndAct
                \end{description}
        \EVT {initiate\_session\_after\_contact\_accept}\cmt{		\\\hspace*{8 cm}  (cf. 3.5.3.4 b) / (cf. 3.5.3.5.2) }
        \EXTD {initiate\_session\_after\_contact}
                \begin{description}
                \AnyPrm
                        \begin{description}
                        \ItemX{l\_partner }
                        \end{description}
                \WhereGrd
                        \begin{description}
                        \nItemXY{ grd2 }{ l\_partner \in  contacted\_by }{ }
                        \nItemY{ grd3 }{ l\_partner \notin  terminated\_ER\_connections }{ }
                        \end{description}
                \ThenAct
                        \begin{description}
                        \nItemXY{ act1 }{ contacted :=  contacted \bunion  \{ l\_partner\}  }{  }
                        \nItemXY{ act2 }{ contacted\_by :=  contacted\_by \setminus  \{ l\_partner\}  }{  }
                        \nItem{ act3 }{ establish\_ER\_connection :=  establish\_ER\_connection \bunion  \{ l\_partner\}  }
                        \end{description}
                \EndAct
                \end{description}
        \EVT {initiate\_session\_no\_contact\_SOM\_accept}\cmt{		\\\hspace*{8,2 cm}  no contact order, i.e., one ofthe other cases of 3.5.3.4 }
        \EXTD {initiate\_session\_no\_contact\_SOM\_accept}
                \begin{description}
                \AnyPrm
                        \begin{description}
                        \ItemXY{l\_partner }{ }
                        \end{description}
                \WhereGrd
                        \begin{description}
                        \nItemX{ grd5 }{ l\_partner \notin  incoming\_sessions \bunion  outgoing\_sessions \bunion  contacted \bunion  contacted\_by }
                        \nItemXY{ grd3 }{ l\_partner \in  RIU \bunion  (RBC \binter  accepting) }{ }
                        \nItemXY{ grd6 }{ current\_status = SOM }{ }
                        \nItemX{ grd7 }{ current\_level \in  \{ L2, L3\}  }
                        \nItem{ grd8 }{ l\_partner \notin  terminated\_ER\_connections }
                        \end{description}
                \ThenAct
                        \begin{description}
                        \nItemXY{ act2 }{ contacted :=  contacted \bunion  \{ l\_partner\}  }{  }
                        \nItem{ act3 }{ establish\_ER\_connection :=  establish\_ER\_connection \bunion  \{ l\_partner\}  }
                        \end{description}
                \EndAct
                \end{description}
        \EVT {initiate\_session\_no\_contact\_SOM\_non\_accept}
        \EXTD {initiate\_session\_no\_contact\_SOM\_non\_accept}
                \begin{description}
                \AnyPrm
                        \begin{description}
                        \ItemXY{l\_partner }{ }
                        \end{description}
                \WhereGrd
                        \begin{description}
                        \nItemX{ grd5 }{ l\_partner \notin  incoming\_sessions \bunion  outgoing\_sessions \bunion  contacted \bunion  contacted\_by }
                        \nItemXY{ grd3 }{ l\_partner \in  RIU \bunion  (RBC \setminus  accepting) }{ }
                        \nItemX{ grd6 }{ current\_status = SOM }
                        \nItemX{ grd7 }{ current\_level \in  \{ L2, L3\}  }
                        \nItem{ grd8 }{ l\_partner \notin  terminated\_ER\_connections }
                        \end{description}
                \ThenAct
                        \begin{description}
                        \nItemXY{ act2 }{ contacted :=  contacted \bunion  \{ l\_partner\}  }{  }
                        \nItemX{ act3 }{ terminating\_sessions :=  terminating\_sessions \bunion  (RBC \binter  (incoming\_sessions \bunion  outgoing\_sessions)) }
                        \nItem{ act4 }{ establish\_ER\_connection :=  establish\_ER\_connection \bunion  \{ l\_partner\}  }
                        \end{description}
                \EndAct
                \end{description}
        \EVT {initiate\_session\_no\_contact\_mode\_change\_accept}
        \EXTD {initiate\_session\_no\_contact\_mode\_change\_accept}
                \begin{description}
                \AnyPrm
                        \begin{description}
                        \ItemXY{l\_partner }{ }
                        \end{description}
                \WhereGrd
                        \begin{description}
                        \nItemX{ grd5 }{ l\_partner \notin  incoming\_sessions \bunion  outgoing\_sessions \bunion  contacted \bunion  contacted\_by }
                        \nItemXY{ grd3 }{ l\_partner \in  RIU \bunion  (RBC \binter  accepting) }{ }
                        \nItemX{ grd6 }{ current\_level \in  \{ L2, L3\}  }
                        \nItemXY{ grd7 }{ signal\_mode\_change = TRUE }{ }
                        \nItemX{ grd8 }{ current\_status \neq  EOM }
                        \nItem{ grd9 }{ l\_partner \notin  terminated\_ER\_connections }
                        \end{description}
                \ThenAct
                        \begin{description}
                        \nItemXY{ act2 }{ contacted :=  contacted \bunion  \{ l\_partner\}  }{  }
                        \nItemX{ act3 }{ signal\_mode\_change :=  FALSE }
                        \nItem{ act4 }{ establish\_ER\_connection :=  establish\_ER\_connection \bunion  \{ l\_partner\}  }
                        \end{description}
                \EndAct
                \end{description}
        \EVT {initiate\_session\_no\_contact\_mode\_change\_non\_accept}
        \EXTD {initiate\_session\_no\_contact\_mode\_change\_non\_accept}
                \begin{description}
                \AnyPrm
                        \begin{description}
                        \ItemXY{l\_partner }{ }
                        \end{description}
                \WhereGrd
                        \begin{description}
                        \nItemX{ grd5 }{ l\_partner \notin  incoming\_sessions \bunion  outgoing\_sessions \bunion  contacted \bunion  contacted\_by }
                        \nItemXY{ grd3 }{ l\_partner \in  RIU \bunion  (RBC \setminus  accepting) }{ }
                        \nItemX{ grd6 }{ current\_level \in  \{ L2, L3\}  }
                        \nItemX{ grd7 }{ signal\_mode\_change = TRUE }
                        \nItemX{ grd8 }{ current\_status \neq  EOM }
                        \nItem{ grd9 }{ l\_partner \notin  terminated\_ER\_connections }
                        \end{description}
                \ThenAct
                        \begin{description}
                        \nItemXY{ act2 }{ contacted :=  contacted \bunion  \{ l\_partner\}  }{  }
                        \nItemX{ act3 }{ terminating\_sessions :=  terminating\_sessions \bunion  (RBC \binter  (incoming\_sessions \bunion  outgoing\_sessions)) }
                        \nItem{ act4 }{ establish\_ER\_connection :=  establish\_ER\_connection \bunion  \{ l\_partner\}  }
                        \end{description}
                \EndAct
                \end{description}
        \EVT {initiate\_session\_no\_contact\_manual\_change\_accept}
        \EXTD {initiate\_session\_no\_contact\_manual\_change\_accept}
                \begin{description}
                \AnyPrm
                        \begin{description}
                        \ItemXY{l\_partner }{ }
                        \end{description}
                \WhereGrd
                        \begin{description}
                        \nItemX{ grd5 }{ l\_partner \notin  incoming\_sessions \bunion  outgoing\_sessions \bunion  contacted \bunion  contacted\_by }
                        \nItemXY{ grd3 }{ l\_partner \in  RIU \bunion  (RBC \binter  accepting) }{ }
                        \nItemX{ grd6 }{ current\_level \in  \{ L2, L3\}  }
                        \nItemX{ grd7 }{ signal\_manual\_level\_change = TRUE }
                        \nItem{ grd8 }{ l\_partner \notin  terminated\_ER\_connections }
                        \end{description}
                \ThenAct
                        \begin{description}
                        \nItemXY{ act2 }{ contacted :=  contacted \bunion  \{ l\_partner\}  }{  }
                        \nItemX{ act3 }{ signal\_manual\_level\_change :=  FALSE }
                        \nItem{ act4 }{ establish\_ER\_connection :=  establish\_ER\_connection \bunion  \{ l\_partner\}  }
                        \end{description}
                \EndAct
                \end{description}
        \EVT {initiate\_session\_no\_contact\_manual\_change\_non\_accept}
        \EXTD {initiate\_session\_no\_contact\_manual\_change\_non\_accept}
                \begin{description}
                \AnyPrm
                        \begin{description}
                        \ItemXY{l\_partner }{ }
                        \end{description}
                \WhereGrd
                        \begin{description}
                        \nItemX{ grd5 }{ l\_partner \notin  incoming\_sessions \bunion  outgoing\_sessions \bunion  contacted \bunion  contacted\_by }
                        \nItemXY{ grd3 }{ l\_partner \in  RIU \bunion  (RBC \setminus  accepting) }{ }
                        \nItemX{ grd6 }{ current\_level \in  \{ L2, L3\}  }
                        \nItemX{ grd7 }{ signal\_manual\_level\_change = TRUE }
                        \nItem{ grd8 }{ l\_partner \notin  terminated\_ER\_connections }
                        \end{description}
                \ThenAct
                        \begin{description}
                        \nItemXY{ act2 }{ contacted :=  contacted \bunion  \{ l\_partner\}  }{  }
                        \nItemX{ act3 }{ terminating\_sessions :=  terminating\_sessions \bunion  (RBC \binter  (incoming\_sessions \bunion  outgoing\_sessions)) }
                        \nItem{ act4 }{ establish\_ER\_connection :=  establish\_ER\_connection \bunion  \{ l\_partner\}  }
                        \end{description}
                \EndAct
                \end{description}
        \EVT {initiate\_session\_no\_contact\_leave\_radio\_hole\_accept}
        \EXTD {initiate\_session\_no\_contact\_leave\_radio\_hole\_accept}
                \begin{description}
                \AnyPrm
                        \begin{description}
                        \ItemXY{l\_partner }{ }
                        \end{description}
                \WhereGrd
                        \begin{description}
                        \nItemX{ grd5 }{ l\_partner \notin  incoming\_sessions \bunion  outgoing\_sessions \bunion  contacted \bunion  contacted\_by }
                        \nItemXY{ grd3 }{ l\_partner \in  RIU \bunion  (RBC \binter  accepting) }{ }
                        \nItemX{ grd6 }{ position\_radio\_hole = FALSE }
                        \nItemX{ grd7 }{ signal\_radio\_hole = TRUE }
                        \nItem{ grd8 }{ l\_partner \notin  terminated\_ER\_connections }
                        \end{description}
                \ThenAct
                        \begin{description}
                        \nItemXY{ act2 }{ contacted :=  contacted \bunion  \{ l\_partner\}  }{  }
                        \nItemX{ act3 }{ signal\_radio\_hole :=  FALSE }
                        \nItem{ act4 }{ establish\_ER\_connection :=  establish\_ER\_connection \bunion  \{ l\_partner\}  }
                        \end{description}
                \EndAct
                \end{description}
        \EVT {initiate\_session\_no\_contact\_leave\_radio\_hole\_non\_accept}
        \EXTD {initiate\_session\_no\_contact\_leave\_radio\_hole\_non\_accept}
                \begin{description}
                \AnyPrm
                        \begin{description}
                        \ItemXY{l\_partner }{ }
                        \end{description}
                \WhereGrd
                        \begin{description}
                        \nItemX{ grd5 }{ l\_partner \notin  incoming\_sessions \bunion  outgoing\_sessions \bunion  contacted \bunion  contacted\_by }
                        \nItemXY{ grd3 }{ l\_partner \in  RIU \bunion  (RBC \setminus  accepting) }{ }
                        \nItemX{ grd6 }{ position\_radio\_hole = FALSE }
                        \nItemX{ grd7 }{ signal\_radio\_hole = TRUE }
                        \nItem{ grd8 }{ l\_partner \notin  terminated\_ER\_connections }
                        \end{description}
                \ThenAct
                        \begin{description}
                        \nItemXY{ act2 }{ contacted :=  contacted \bunion  \{ l\_partner\}  }{  }
                        \nItemX{ act3 }{ terminating\_sessions :=  terminating\_sessions \bunion  (RBC \binter  (incoming\_sessions \bunion  outgoing\_sessions)) }
                        \nItem{ act4 }{ establish\_ER\_connection :=  establish\_ER\_connection \bunion  \{ l\_partner\}  }
                        \end{description}
                \EndAct
                \end{description}
        \EVT {initiate\_session\_after\_timeout\_accept}
        \EXTD {initiate\_session\_no\_contact\_accept}
                \begin{description}
                \AnyPrm
                        \begin{description}
                        \ItemXY{l\_partner }{ }
                        \end{description}
                \WhereGrd
                        \begin{description}
                        \nItemX{ grd5 }{ l\_partner \notin  incoming\_sessions \bunion  outgoing\_sessions \bunion  contacted \bunion  contacted\_by }
                        \nItemXY{ grd3 }{ l\_partner \in  RIU \bunion  (RBC \binter  accepting) }{ }
                        \nItem{ grd6 }{ l\_partner \in  terminated\_ER\_connections }
                        \end{description}
                \ThenAct
                        \begin{description}
                        \nItemXY{ act2 }{ contacted :=  contacted \bunion  \{ l\_partner\}  }{  }
                        \nItem{ act3 }{ terminated\_ER\_connections :=  terminated\_ER\_connections \setminus  \{ l\_partner\}  }
                        \nItem{ act4 }{ establish\_ER\_connection :=  establish\_ER\_connection \bunion  \{ l\_partner\}  }
                        \end{description}
                \EndAct
                \end{description}
        \EVT {initiate\_session\_after\_timeout\_non\_accept}
        \EXTD {initiate\_session\_no\_contact\_non\_accept}
                \begin{description}
                \AnyPrm
                        \begin{description}
                        \ItemXY{l\_partner }{ }
                        \end{description}
                \WhereGrd
                        \begin{description}
                        \nItemX{ grd5 }{ l\_partner \notin  incoming\_sessions \bunion  outgoing\_sessions \bunion  contacted \bunion  contacted\_by }
                        \nItemXY{ grd3 }{ l\_partner \in  RIU \bunion  (RBC \setminus  accepting) }{ }
                        \nItem{ grd6 }{ l\_partner \in  terminated\_ER\_connections }
                        \end{description}
                \ThenAct
                        \begin{description}
                        \nItemXY{ act2 }{ contacted :=  contacted \bunion  \{ l\_partner\}  }{  }
                        \nItemX{ act3 }{ terminating\_sessions :=  terminating\_sessions \bunion  (RBC \binter  (incoming\_sessions \bunion  outgoing\_sessions)) }
                        \nItem{ act4 }{ terminated\_ER\_connections :=  terminated\_ER\_connections \setminus  \{ l\_partner\}  }
                        \nItem{ act5 }{ establish\_ER\_connection :=  establish\_ER\_connection \bunion  \{ l\_partner\}  }
                        \end{description}
                \EndAct
                \end{description}
        \EVT {establish\_ER\_connection\_SOM}
                \begin{description}
                \AnyPrm
                        \begin{description}
                        \Item{l\_partner }
                        \end{description}
                \WhereGrd
                        \begin{description}
                        \nItem{ grd1 }{ l\_partner \in  contacted }
                        \nItemY{ grd2 }{ l\_partner \in  establish\_ER\_connection }{ }
                        \nItem{ grd3 }{ current\_status = SOM }
                        \end{description}
                \ThenAct
                        \begin{description}
                        \nItem{ act1 }{ establish\_ER\_connection :=  establish\_ER\_connection \setminus  \{ l\_partner\}  }
                        \nItem{ act2 }{ ER\_connections :=  ER\_connections \bunion  \{ l\_partner\}  }
                        \end{description}
                \EndAct
                \end{description}
        \EVT {establish\_ER\_connection}
                \begin{description}
                \AnyPrm
                        \begin{description}
                        \ItemY{l\_partner }{ }
                        \end{description}
                \WhereGrd
                        \begin{description}
                        \nItem{ grd1 }{ l\_partner \in  contacted }
                        \nItem{ grd2 }{ l\_partner \in  establish\_ER\_connection }
                        \nItemY{ grd3 }{ current\_status \neq  SOM }{ }
                        \end{description}
                \ThenAct
                        \begin{description}
                        \nItem{ act1 }{ establish\_ER\_connection :=  establish\_ER\_connection \setminus  \{ l\_partner\}  }
                        \nItem{ act2 }{ ER\_connections :=  ER\_connections \bunion  \{ l\_partner\}  }
                        \end{description}
                \EndAct
                \end{description}
        \EVT {est\_perform\_end\_of\_mission}\cmt{}
                \begin{description}
                \AnyPrm
                        \begin{description}
                        \Item{l\_partner }
                        \end{description}
                \WhereGrd
                        \begin{description}
                        \nItemY{ grd1 }{ l\_partner \in  contacted }{ }
                        \nItem{ grd2 }{ l\_partner \in  establish\_ER\_connection }
                        \nItemY{ grd3 }{ signal\_status\_change = TRUE }{ }
                        \nItem{ grd4 }{ current\_status = EOM }
                        \end{description}
                \ThenAct
                        \begin{description}
                        \nItem{ act1 }{ establish\_ER\_connection :=  establish\_ER\_connection \setminus  \{ l\_partner\}  }
                        \end{description}
                \EndAct
                \end{description}
        \EVT {est\_terminate\_order}
        \EXTD {drop\_contact}
                \begin{description}
                \AnyPrm
                        \begin{description}
                        \ItemX{l\_partner }
                        \end{description}
                \WhereGrd
                        \begin{description}
                        \nItemX{ grd1 }{ l\_partner \in  contacted }
                        \nItemY{ grd2 }{ l\_partner \in  establish\_ER\_connection }{ }
                        \nItem{ grd3 }{ current\_status \neq  SOM }
                        \end{description}
                \ThenAct
                        \begin{description}
                        \nItemX{ act1 }{ contacted :=  contacted \setminus  \{ l\_partner\}  }
                        \nItem{ act2 }{ establish\_ER\_connection :=  establish\_ER\_connection \setminus  \{ l\_partner\}  }
                        \end{description}
                \EndAct
                \end{description}
        \EVT {est\_pass\_level\_transition}\cmt{ }
                \begin{description}
                \AnyPrm
                        \begin{description}
                        \Item{l\_partner }
                        \end{description}
                \WhereGrd
                        \begin{description}
                        \nItem{ grd1 }{ l\_partner \in  contacted }
                        \nItem{ grd2 }{ l\_partner \in  establish\_ER\_connection }
                        \nItem{ grd3 }{ signal\_level\_change = TRUE }
                        \nItem{ grd4 }{ current\_status \neq  SOM }
                        \nItem{ grd5 }{ current\_level \in  \{ L0, L1, NTC\}  }
                        \end{description}
                \ThenAct
                        \begin{description}
                        \nItem{ act1 }{ establish\_ER\_connection :=  establish\_ER\_connection \setminus  \{ l\_partner\}  }
                        \end{description}
                \EndAct
                \end{description}
        \EVT {est\_pass\_radio\_hole}
                \begin{description}
                \AnyPrm
                        \begin{description}
                        \Item{l\_partner }
                        \end{description}
                \WhereGrd
                        \begin{description}
                        \nItem{ grd1 }{ l\_partner \in  contacted }
                        \nItem{ grd2 }{ l\_partner \in  establish\_ER\_connection }
                        \nItem{ grd3 }{ signal\_radio\_hole = TRUE \land  position\_radio\_hole = TRUE }
                        \nItem{ grd4 }{ current\_status \neq  SOM }
                        \end{description}
                \ThenAct
                        \begin{description}
                        \nItem{ act1 }{ establish\_ER\_connection :=  establish\_ER\_connection \setminus  \{ l\_partner\}  }
                        \end{description}
                \EndAct
                \end{description}
        \EVT {est\_RIU\_leave\_L1}
                \begin{description}
                \AnyPrm
                        \begin{description}
                        \Item{l\_partner }
                        \end{description}
                \WhereGrd
                        \begin{description}
                        \nItem{ grd1 }{ l\_partner \in  contacted }
                        \nItem{ grd2 }{ l\_partner \in  RIU }
                        \nItem{ grd3 }{ signal\_level\_change = TRUE }
                        \nItem{ grd5 }{ current\_level \neq  L1 }
                        \nItem{ grd4 }{ current\_status \neq  SOM }
                        \end{description}
                \ThenAct
                        \begin{description}
                        \nItem{ act1 }{ establish\_ER\_connection :=  establish\_ER\_connection \setminus  \{ l\_partner\}  }
                        \end{description}
                \EndAct
                \end{description}
        \EVT {est\_RBC\_border}
                \begin{description}
                \AnyPrm
                        \begin{description}
                        \Item{l\_partner }
                        \end{description}
                \WhereGrd
                        \begin{description}
                        \nItem{ grd1 }{ l\_partner \in  contacted }
                        \nItem{ grd2 }{ l\_partner \in  RBC }
                        \nItemY{ grd3 }{ signal\_RBC\_border = TRUE }{ }
                        \nItem{ grd4 }{ current\_status \neq  SOM }
                        \end{description}
                \ThenAct
                        \begin{description}
                        \nItem{ act1 }{ establish\_ER\_connection :=  establish\_ER\_connection \setminus  \{ l\_partner\}  }
                        \end{description}
                \EndAct
                \end{description}
        \EVT {indicate\_RBC\_border}
                \begin{description}
                \AnyPrm
                        \begin{description}
                        \ItemY{l\_flag }{ }
                        \end{description}
                \WhereGrd
                        \begin{description}
                        \nItem{ grd1 }{ l\_flag \in  BOOL }
                        \end{description}
                \ThenAct
                        \begin{description}
                        \nItemY{ act1 }{ signal\_RBC\_border :=  l\_flag }{  }
                        \end{description}
                \EndAct
                \end{description}
        \EVT {est\_other\_RBC\_non\_accept}
                \begin{description}
                \AnyPrm
                        \begin{description}
                        \ItemY{l\_partner }{ }
                        \end{description}
                \WhereGrd
                        \begin{description}
                        \nItem{ grd1 }{ l\_partner \in  contacted }
                        \nItem{ grd2 }{ l\_partner \in  RBC }
                        \nItem{ grd3 }{ RBC \binter  accepting \binter  contacted\_by \neq  \emptyset  }
                        \nItem{ grd4 }{ current\_status \neq  SOM }
                        \end{description}
                \ThenAct
                        \begin{description}
                        \nItem{ act1 }{ establish\_ER\_connection :=  establish\_ER\_connection \setminus  \{ l\_partner\}  }
                        \end{description}
                \EndAct
                \end{description}
        \EVT {timeout\_ER\_connection}
        \EXTD {drop\_session}
                \begin{description}
                \AnyPrm
                        \begin{description}
                        \ItemX{l\_partner }
                        \end{description}
                \WhereGrd
                        \begin{description}
                        \nItemX{ grd1 }{ l\_partner \in  incoming\_sessions \bunion  outgoing\_sessions }
                        \nItem{ grd3 }{ l\_partner \in  ER\_connections }
                        \end{description}
                \ThenAct
                        \begin{description}
                        \nItemXY{ act1 }{ incoming\_sessions :=  incoming\_sessions \setminus  \{ l\_partner\}  }{  }
                        \nItemX{ act2 }{ outgoing\_sessions :=  outgoing\_sessions \setminus  \{ l\_partner\}  }
                        \nItem{ act3 }{ ER\_connections :=  ER\_connections \setminus  \{ l\_partner\}  }
                        \nItem{ act4 }{ terminated\_ER\_connections :=  terminated\_ER\_connections \bunion  \{ l\_partner\}  }
                        \end{description}
                \EndAct
                \end{description}
        \EVT {terminate\_communication}
        \EXTD {terminate\_communication}
                \begin{description}
                \AnyPrm
                        \begin{description}
                        \ItemX{l\_partner }
                        \end{description}
                \WhereGrd
                        \begin{description}
                        \nItemX{ grd1 }{ l\_partner \in  incoming\_sessions \bunion  outgoing\_sessions }
                        \nItemX{ grd2 }{ l\_partner \in  terminating\_sessions }
                        \nItem{ grd3 }{ l\_partner \notin  terminated\_ER\_connections }
                        \end{description}
                \ThenAct
                        \begin{description}
                        \nItemX{ act1 }{ incoming\_sessions :=  incoming\_sessions \setminus  \{ l\_partner\}  }
                        \nItemX{ act2 }{ outgoing\_sessions :=  outgoing\_sessions \setminus  \{ l\_partner\}  }
                        \nItemX{ act3 }{ terminating\_sessions :=  terminating\_sessions \setminus  \{ l\_partner\}  }
                        \nItem{ act4 }{ ER\_connections :=  ER\_connections \setminus  \{ l\_partner\}  }
                        \end{description}
                \EndAct
                \end{description}
        \EVT {make\_RBC\_accepting}
        \EXTD {make\_RBC\_accepting}
                \begin{description}
                \AnyPrm
                        \begin{description}
                        \ItemXY{l\_partner }{ }
                        \end{description}
                \WhereGrd
                        \begin{description}
                        \nItemX{ grd1 }{ l\_partner \in  RBC }
                        \end{description}
                \ThenAct
                        \begin{description}
                        \nItemX{ act1 }{ accepting :=  accepting \bunion  \{ l\_partner\}  }
                        \end{description}
                \EndAct
                \end{description}
        \EVT {make\_RBC\_non\_accepting}
        \EXTD {make\_RBC\_non\_accepting}
                \begin{description}
                \AnyPrm
                        \begin{description}
                        \ItemX{l\_partner }
                        \end{description}
                \WhereGrd
                        \begin{description}
                        \nItemX{ grd1 }{ l\_partner \in  accepting }
                        \end{description}
                \ThenAct
                        \begin{description}
                        \nItemX{ act1 }{ accepting :=  accepting \setminus  \{ l\_partner\}  }
                        \end{description}
                \EndAct
                \end{description}
\END
\end{description}


\bibliographystyle{alpha}
\bibliography{openetcs}

\end{document}


%%% Local Variables:
%%% mode: latex
%%% TeX-master: t
%%% End:
