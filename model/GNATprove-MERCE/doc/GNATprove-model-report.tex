\documentclass{template/openetcs_report}
% Use the option "nocc" if the document is not licensed under Creative Commons
%\documentclass[nocc]{template/openetcs_report}
\usepackage{url}
\usepackage{hyperref}
\usepackage{listings}
\graphicspath{{./template/}{.}{./images/}}

% listings package configuration
\lstset{basicstyle={\footnotesize \sffamily},framesep=2pt,frame=single}
\newcommand{\Ada}[1]{\lstinline[language=Ada,basicstyle={\sffamily},framesep=0pt]{#1}}

\begin{document}
\frontmatter
\project{openETCS}

%Please do not change anything above this line
%============================
% The document metadata is defined below

%assign a report number here
\reportnum{OETCS/WP7/D??}

%define your workpackage or task here
\wp{openETCS@ITEA Work Package 7: ``Toolchain''}

%set a title here
\title{Evaluation model of ETCS using GNATprove}

%set a subtitle here
\subtitle{Tool and model presentation}

%set the date of the report here
\date{March 2013}

%define a list of authors and their affiliation here
\author{David Mentré}

\affiliation{Mitsubishi Electric R\&D Centre Europe\\
1 allée de Beaulieu\\
CS 10806\\
35708 RENNES cedex 7\\
\\
email: \url{d.mentre@fr.merce.mee.com}
}

% define the coverart
\coverart[width=350pt]{openETCS_EUPL.png}

%define the type of report
\reporttype{Draft Report}


\begin{abstract}
%define an abstract here
  FIXME
\end{abstract}

%=============================
%Do not change the next three lines
\maketitle
\tableofcontents
\listoffiguresandtables
%=============================
% The actual document starts below this line
%=============================

%Start here

\mainmatter

\chapter{Introduction}

This document presents the use of Ada 2012 language and GNATprove tool
to model a subset of ETCS. This subset is included in the subset
defined in openETCS D2.5 document\cite{openetcs:D2.5}.

In this report, we firstly present the Ada 2012 language and GNATprove
tool, then we give an overview of our model, its benefits and
shortcoming to end with the detailed source code of the complete
model.

\chapter{Short introduction to Ada 2012 and GNATprove tool}

This model is using Ada 2012 language\cite{arm2012}. The Ada language
is a well known generic purpose language which first version was
published in 1983 and which is normalized by ISO. After several
revisions in 1995 and 2005, the latest 2012 revision offers
interesting facilities for program verification. More specifically,
function contracts can now be declared through pre and post-conditions
using \Ada{Pre} and \Ada{Post} Ada annotations.

Those contracts can be compiled in executable code and checked
dynamically at execution, or statically verified using a dedicated
tool. GNATprove is such a static verification tool. It
\emph{automatically}\footnote{Automatic verification is made through
  several calls to SMT (Satisfiability Modulo Theories) solver
  \emph{Alt-Ergo}.} checks Ada 2012 contracts and absence of run-time
exception (underflow, overflow, out of bound access, ...) for all
possible executions. GNATprove is integrated into AdaCore's GNAT
Programming Studio (GPS) environment.

GNATprove and GPS programming environment are Open Source software,
licensed under GNU GPL.

\section{A short example}

As example, here is the Ada 2012 specification of a
\Ada{Saturated_Sum} function. The \Ada{Post}-condition states that if
the sum of two integers \Ada{X1} and \Ada{X2} is below a given
\Ada{Maximum}, then this sum is returned, otherwise the \Ada{Maximum}
is returned.

The \Ada{Pre}-condition states that both \Ada{X1} and \Ada{X2} should
be below the biggest \Ada{Natural} divided by 2, such that computation
of the sum does not raise an overflow error at execution.

\lstinputlisting[language=Ada]{../src/example.ads}

The implementation of this specification is rather obvious.

\lstinputlisting[language=Ada]{../src/example.adb}

Applying GNATprove on above two files generates 9 Verification
Conditions (VC) that are all \emph{automatically} proved correct by
the tool.

After such verifications, we are sure that for all possible executions
no run-time errors are going to be raised and that the
\Ada{Saturated_Sum} function will fulfill its contract if called with
pre-conditions satisfied.

\chapter{Model overview}

The model is organized along SUBSET 026: to each section or subsection
corresponds an Ada \Ada{package} of the same name\footnote{We are not
  sure this approach makes the model easily understandable, at least
  without a good knowledge of SUBSET 026. Naming packages after
  described entities (Communication, Balises, etc.) might be a better
  approach for a full fledged model.}. Like Uwe Steinke for its SCADE
model, we have tried to formalize each paragraph of SUBSET 026,
associating to it some Ada 2012 code.

For each part of the model, a comment gives the paragraph number it
corresponds to, thus making a basic traceability.

We have also created additional packages for generic parts of the
model or to define data structure used by several parts.

We have created Ada data types to describe specific objects of ETCS
specification. Those data types are in turned used to define Ada
entities that represent ETCS specification objects in our model.

For example, the following \Ada{package ETCS_Level} defines the five
ETCS levels 0 to 4 as \Ada{type ertms_etcs_level_t}. Within those
levels, the NTC (aka STM) specific level is defined as number 4 with
\Ada{ertms_etcs_level_ntc constant}. Then this data type is used for
definition of \Ada{ertms_etcs_level} variable that describes current
ETCS level in the model.

\begin{lstlisting}[language=Ada]
Package ETCS_Level is
   type ertms_etcs_level_t is range 0..4; -- SUBSET-026-2.6.2.3
   ertms_etcs_level_ntc : constant ertms_etcs_level_t := 4;

   ertms_etcs_level : ertms_etcs_level_t;
end;
\end{lstlisting}

For propositions in the text, e.g. ``Start of Mission'', we have used
a \Ada{Boolean} variable, e.g. \Ada{Start_Of_Mission}.

\chapter{Model benefits and shortcomings}

FIXME

\chapter{Detailed model description}

FIXME

\section{Generic parts of the model}

\lstinputlisting[language=Ada]{../src/data_types.ads}
\lstinputlisting[language=Ada]{../src/etcs_level.ads}
\lstinputlisting[language=Ada]{../src/appendix_a_3_1.ads}


\subsection{SUBSET-026-4.3.2 ETCS modes}

\lstinputlisting[language=Ada]{../src/section_4_3_2.ads}

\section{SUBSET-026-4.6 Transitions between modes}

\lstinputlisting[language=Ada]{../src/section_4_6.ads}
\lstinputlisting[language=Ada]{../src/section_4_6.adb}


\section{SUBSET-026-3.5.3 Establishing a communication session}

\subsection{Safe Radio package}

\lstinputlisting[language=Ada]{../src/safe_radio.ads}
\lstinputlisting[language=Ada]{../src/safe_radio.adb}

\subsection{Com map utility package}

\lstinputlisting[language=Ada]{../src/com_map.ads}

\subsection{Section 3.5.3 modeling}

\lstinputlisting[language=Ada]{../src/section_3_5_3.ads}
\lstinputlisting[language=Ada]{../src/section_3_5_3.adb}

\section{SUBSET-026-3.13 Speed and distance monitoring}

\subsection{Modeling of step functions}

\lstinputlisting[language=Ada]{../src/step_function.ads}
\lstinputlisting[language=Ada]{../src/step_function.adb}
\lstinputlisting[language=Ada]{../src/step_function_test.adb}

%% \begin{figure}
%%   \centering
%%   \fbox{\includegraphics[width=2in]{itea}}
%%   \caption{Yet Another Castle In Appendix}
%%   \label{fig:castle2}
%% \end{figure}

\bibliographystyle{abbrv}
\bibliography{GNATprove-model-report}


%===================================================
%Do NOT change anything below this line

\end{document}

% LocalWords:  GNATprove AdaCore's GPS VC pre adb GPL SMT Satisfiability ETCS
% LocalWords:  FIXME
