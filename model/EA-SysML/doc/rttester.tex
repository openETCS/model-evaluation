\subsection{Tool overview}
\begin{itemize}
\item Purpose
\item Features
\item Certification
\end{itemize}
Paper Certification + bases

The RT-Tester follows the model-based testing approach. 
It provides the following features :
\begin{itemize}
\item Automated Test Case Generation 
\item Automated Test Data Generation 
\item Automated Test Procedure Generation 
\item Automated Requirement Tracing 
\item Test Management system 
\end{itemize}
Starting from a test model design with UML/SYML, the RT-tester fully
automatically generates test cases. They are then specified as test data
(sequences of stimuli with timing constraints) and used to stimulate the SUT and
run concurently with the generated test oracles. The test procedure is the
combination of the test oracles and the SUT composed a test porcdure that can be
compiled and executed.

\FIXME{Basic schema}


\subsection{Test generation}
\begin{itemize}
\item Test coverage
\item Requirement coverage
\item LTL requirement
\end{itemize}
