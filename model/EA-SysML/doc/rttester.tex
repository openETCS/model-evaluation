\subsection{Tool overview}

The RT-Tester test automation tool, made by Verified
\cite{verified_website}, performs automatic test generation, test
execution and real-time test evaluation.  It supports different
testing approach such as unit testing, software integration testing
for component, hardware/software integration testing and system
integration testing.  The RT-Tester version we had used, follows the
model-based testing approach \cite{Peleska2011,rttmbtreport2011} and
it provides the following features :
\begin{itemize}
\item Automated Test Case Generation 
\item Automated Test Data Generation 
\item Automated Test Procedure Generation 
\item Automated Requirement Tracing 
\item Test Management system 
\end{itemize}
Starting from a test model design with UML/SYML, the RT-tester fully
automatically generates test cases. They are then specified as test
data (sequences of stimuli with timing constraints) and used to
stimulate the SUT and run concurently with the generated test
oracles. The test procedure is the combination of the test oracles and
the SUT that can be compiled and executed.

The tool supports test cases/data generation for structural
testing. It automatically generates  reach statement coverage, branch coverage and
modified condition/decision coverage (MC/DC) as far as this is possible.
The test cases may all be linked to requirements ensuring a complete
requirement traceability. 



Finally the tool may produce the documentation of tests for
certification purposes. For each test cases the following document are
produced :
\begin{itemize}
\item {\em Test procedure}: that specifies  how one test case can be
  executed, its associated test data produced and how the SUT
  reactions are evaluated against the expected results.
\item {\em Test report}: that summarizes all relevant information
  about the test execution.
\end{itemize}

In \cite{brauer_efficient_2012}, a general approach  on how to qualify
model-based testing tool according to the standard ISO 26262 ad RTCA
DO178C has been proposed and applied with success to the RT-tester
tool. Following the same  approach compatibility with the CENELEC EN50128
may be easily done. 




\subsection{Test generation}
\begin{itemize}
\item Test coverage
\item Requirement coverage
\item LTL requirement
\end{itemize}
\FIXME{Table coverage summary}
 	


%%% Local Variables: 
%%% mode: latex
%%% TeX-master: "descriptionReport"
%%% End: 

